
%% bare_conf.tex
%% V1.4b
%% 2015/08/26
%% by Michael Shell
%% See:
%% http://www.michaelshell.org/
%% for current contact information.
%%
%% This is a skeleton file demonstrating the use of IEEEtran.cls
%% (requires IEEEtran.cls version 1.8b or later) with an IEEE
%% conference paper.
%%
%% Support sites:
%% http://www.michaelshell.org/tex/ieeetran/
%% http://www.ctan.org/pkg/ieeetran
%% and
%% http://www.ieee.org/
%%*************************************************************************
%% Legal Notice:
%% This code is offered as-is without any warranty either expressed or
%% implied; without even the implied warranty of MERCHANTABILITY or
%% FITNESS FOR A PARTICULAR PURPOSE!
%% User assumes all risk.
%% In no event shall the IEEE or any contributor to this code be liable for
%% any damages or losses, including, but not limited to, incidental,
%% consequential, or any other damages, resulting from the use or misuse
%% of any information contained here.
%%
%% All comments are the opinions of their respective authors and are not
%% necessarily endorsed by the IEEE.
%%
%% This work is distributed under the LaTeX Project Public License (LPPL)
%% ( http://www.latex-project.org/ ) version 1.3, and may be freely used,
%% distributed and modified. A copy of the LPPL, version 1.3, is included
%% in the base LaTeX documentation of all distributions of LaTeX released
%% 2003/12/01 or later.
%% Retain all contribution notices and credits.
%% ** Modified files should be clearly indicated as such, including  **
%% ** renaming them and changing author support contact information. **
%%*************************************************************************



\documentclass[a4paper,conference]{IEEEtran}

\def\citepunt{,}

\usepackage[pdftex]{graphicx}
\ifCLASSOPTIONcompsoc
  \usepackage[caption=false,font=normalsize,labelfont=sf,textfont=sf]{subfig}
\else
  \usepackage[caption=false,font=footnotesize]{subfig}
\fi
\usepackage{comment}
\usepackage{float}
\usepackage{mathtools}
\usepackage{amsfonts}
\usepackage{bm}
\usepackage{pgf,tikz,pgfplots}
\usetikzlibrary{calc}

\DeclarePairedDelimiter\ceil{\lceil}{\rceil}
\DeclarePairedDelimiter\floor{\lfloor}{\rfloor}

\DeclareMathOperator{\vect}{vec}

\newcommand{\R}{\mathbb{R}}
\newcommand{\A}{\mathcal{A}}
\newcommand{\D}{\mathcal{D}}
\newcommand{\I}{\hat{I}}
\newcommand{\Z}{\mathbb{Z}}
\newcommand{\m}[1]{{\mathrm{\bf #1}}}
\newcommand{\E}{\tilde{\m{I}}}
\newcommand{\F}{\hat{F}}
\newcommand{\lI}{\m{I}}
\newcommand{\mF}{\m{F}}
\newcommand{\bF}{\mathcal{F}}
\newcommand{\J}{\hat{J}}
\newcommand{\lJ}{\m{J}}
\newcommand{\pD}{D^\prime}
\newcommand{\eF}{\hat{\mF}}
\newcommand{\gN}{\m{N}}
\newcommand{\W}{\m{W}}
\newcommand{\M}{\mathcal{M}}
\newcommand{\Pa}{\mathcal{P}}
\newcommand{\pDD}{\D^\prime}

\definecolor{navy}{RGB}{0,0,137}
\definecolor{tealDeer}{RGB}{148,232,180}
\definecolor{dodgerBlue}{RGB}{18,161,255}
\definecolor{citrine}{RGB}{230, 194, 8}
\definecolor{violet}{RGB}{112,5,164}
\definecolor{navyPurple}{RGB}{172,86,253}
\definecolor{heliotrope}{RGB}{236,93,253}

% correct bad hyphenation here
\hyphenation{op-tical net-works semi-conduc-tor}


\begin{document}

\title{Learning filters from user provided image makers for convolutional neural networks}

% make the title area
\maketitle

\begin{abstract}
  Identifying the species of trees in a region is essential to identify the type of land use that is taking place. It is also essential to assess and monitor plantation performance and analyze the impact of natural disasters. The manual identification of these trees is tedious, costly, and error-prone, so the application of automatic methods is necessary. Deep learning is quite effective in applications from different domains; however, it usually needs a considerable amount of annotated data so that the models' appropriate training can be done. As deep as the model gets, it needs more data. We propose a training method for feature extractors of convolutional neural networks that need only a minimal set of images and that the CNN designer has more control over. The method uses markers that the user places in relevant regions of some training images to learn the convolutional layers' filters allowing the user to interfere with the training process directly. As this method does not rely on backpropagation or minimization of some cost function to learn the feature extractor filters, it does not require many training images. The method was applied to the problem of determining whether area image patches contain coconut trees or not. The model trained with this method achieved competitive results with state of the art.
\end{abstract}

\section{Introduction}
Deep learning has proven to be applicable to different types of tasks, from image classification to the creation of synthetic data \cite{goodfellow2016deep}. Given the success of the application of deep learning in various areas of knowledge, this technique has also been applied to several tasks of remote sensing. Examples of this technique exist in various problems such as segmentation of terrain images \cite{kemker2018algorithms, kampffmeyer2016semantic, hamaguchi2018effective}, building identification \cite{xu2018building, lu2018detecting, liu2018multilevel}, and deforestation monitoring \cite{bragilevsky2017deep}. 

The detection of trees is important to produce knowledge about the land use of a region, from the assessment of crop diversity or the performance of plantations in a given season to the assessment of damages after natural disasters such as drilling and flooding. As these plantations can span very large areas, manual detection can be costly and susceptible to errors. 

In \cite{fassnacht2016review}, we can see a review of the problem of classification of tree species of aerial images. \cite{puttemans2018comparing} apply deep learning to the problem of detecting coconut trees in aerial images. Aparta et al. \cite{aparna2018cnn} apply convolutional neural networks to count coconut trees on plantations in India. Vargas-Muñoz et al. \cite{8899005} propose an active learning approach to detect coconut trees that seeks to reduce the effort of annotating images. The \cite{8899005} approach is divided into a stage of producing candidate patches for having a coconut tree and a stage where a CNN determines whether a patch has a coconut tree. We propose a different way to produce this classifier.  

Despite this success, methods based on deep learning require a lot of annotated images for training, and annotating this volume of data is quite costly. Even interactive methods require large training sets. If these methods depend on huge training sets, their application may not be feasible in areas where it is impossible, or the cost of gathering and annotating these data sets is prohibitive. Much of the user's interaction with methods based on deep learning occurs through the annotation of the training set, and the user does not have much influence on the training process.

Despite the advances in deep learning over the last decades, especially with most recent deep neural networks, some important questions related to human interaction remain unanswered: (1) How to choose the most reasonable model, in the sense that it provides acceptable effectiveness, being as simple and efficient as needed, for a given classification problem? (2) How to train that model from a reduced number of samples?  (3) Can the user explain the decisions of that model? (4) Can that model improve from human corrections? The first question requires to explore human knowledge about the classification technique and the problem of interest. The second one requires to reduce as much as possible the human effort to train a model. The third issue is related to human understanding, and it may explore techniques (such as data visualization) to explain the decisions of the model and perhaps to guide human decisions concerning the design of the model. The fourth question is also important during training, and it is related to human control over the process. They all lead to the importance of involving human experts during the machine learning process.

In this paper, we present some advances towards the answers to the questions (1) and (2). We also discuss future research related to the questions (3) and (4). We present a method to the design of light-weighted convolutional neural networks from markers drawn by an expert on a few images to indicate representative regions of each object of interest for classification. From those markers, our approach can learn the filters of each layer, and the convolutional neural network can be designed in a layer-by-layer fashion, with no need for backpropagation in order to build a representative feature space for the classification problem. Backpropagation is only used to train a bank of multi-layer perceptrons for local pattern classification.

We propose a new way of learning the filters of the convolutional layers of deep neural networks that do not depend on the optimization of an error function and, therefore, do not need a large number of training samples. The training process is done with user intervention, which adds markers to the training images to indicate relevant features to characterize each class. To learn the filters, we get a patch around each marker of a class and cluster them. The centroid of each cluster is a filter that can be used to identify features of this cluster. The set of centroids of clusters of all clusters is the convolutional layer filters. For the next layer, the patches are generated on the output of the previous one. To evaluate our method, we use a dataset of aerial images to detect coconut trees. We identified that, with few layers and few training samples, we managed to get competitive results with state-of-the-art models.

\section{Learning filters from markers}
\label{sec:method}

Convolutional neural networks perform well in image classification tasks due to its capability of extracting features. That is, convolutional layers produce great descriptors that can be used by the classifier. Conventionally, during the training process, an algorithm tries to minimize a loss function that measures how good the classifier is performing with the features given by the feature extractor. The algorithm of optimization tries through backpropagation to find out which features to extract. Our method is based on the idea that the network designer is an expert in the application domain or knows well enough to indicate which are the relevant characteristics of the objects of interest. Hence, it does not rely on the optimization of some loss function to learn filters for those relevant characteristics. In our approach, the network designer highlights features that she believes are the ones that characterize each class by placing markers. Those markers are used to learn the filters to identify the features they represent.

Let $\D$ be a dataset of images, and let $I \in \D$ be an image with dimensions $X \times Y$. A marker $m$ is associated with a pixel $p(m) = (x, y)$ and has a label $L(m) = c$. Let $M(I)$ be the set of markers of image $I$.  For each maker $m \in M(I)$, we create a patch $P$ with dimensions $k \times k$ from a window centered in the marker's pixel as $P(I, x ,y) = I(x+i, y+j)$ for $i, j = -\floor{k/2}, \ldots, \floor{k/2}$.We pad images with $0$ so that we can create patches from border pixels. For each class, we create a set of patches made using the markers of that class. Let \[\mathcal{P}_c = \bigcup_{ I \in \D, m \in M(I) :\, L(m) = c}{P(I, p(m))}\] be the set of all patches made from markers of class $c$.

Given set of patches $\Pa_c$, we can use a clustering algorithm to discover groups in $\Pa_c$. Each group represents local patterns of class $c$, highlighted by the designer through the markers. Since the centroid of each cluster must be a good representative of the characteristics of the group, we can use them as filters to enhance those patterns in the images of $\D$. We must do this for each class.

However, for it to work properly, patches and images need to be centered at the origin. For this, we use the information from the patches of all classes to calculate the mean and standard deviation per band. If the patches are representative of images in $\D$, the mean and standard deviation per band of $\Pa$ are close to the mean and standard deviation per band in $\D$.

\begin{comment}
\begin{figure}[t]
  \begin{center}
     \includegraphics[width=0.8\linewidth]{figures/filter.png}
  \end{center}
  \caption{Filter learned from clustering.}
  \label{fig:filter}
\end{figure}
\end{comment}


Let $F$ be a filter with dimension $k \times k$, and let $\pi$ be the patch around pixel $(x,y)$ also with dimension $k \times k$. By the definition of convolution in deep learning, we can interpret this operation for a pixel $(x, y)$ as the dot product between the kernel vector $\vect(F)$ and the vector of a patch $\vect(\pi)$ around this pixel, that is, the convolution in $(x, y)$ is the same as $\vect(F) \cdot \vect(\pi)$. Thus, let $H$ be a hyperplane orthogonal to vector $\vect(F)$. Vectors similar to $\vect(F)$ must fall on the same side of the hyperplane as $\vect(F)$. Therefore, if vector $\vect(\pi)$ is on the same side of $H$ as $\vect(F)$, then the doct product between $\vect(\pi)$ and $\vect(F)$ is positive, and it is negative otherwise. Thus, the doct product between the centroid of a cluster and each of the vectors of that cluster must be positive, which implies that the centroid is a good filter to identify the pattern described by that cluster. Hence, if we throw away the negative values, applying convolution with centroids as filters must highlight regions that has a certain pattern that the network designer believes is important.

Since all the patches used to produce the filters are centered around the origin, as well as the images of $\D$, we have that the hyperplanes ortogonal to the filters contain the origin, therefor there is no need for bias.

Each convolutional layer is trained individually, one layer at a time. After the convolution operation, we apply the ReLU function to eliminate negative activations and the max pooling operation to aggregate local information. We preserve the output dimensions of convolutional layers so that we can use the same markers used in the previous layer. The new patches are generated from the previeus convolutional layer output and the whole process is repeated to find the filters of the next one.

\begin{figure}[!t]
  \centering
  \subfloat[\label{fig:ex-groups}]
  {\includegraphics[width=.35\linewidth]{figures/example_groups/groups_centered}}
  ~
  \subfloat[\label{fig:ex-groups-after}]{\includegraphics[width=.45\linewidth]{figures/example_groups/groups3d_centered}
  }
  \caption{Exemple of application of convolution with groups centers as filters' weightes: (a) Groups of three classes centered on the origin; (b) Groups after applying convolution and activation operations. }
  \label{fig:filter}
\end{figure}

\begin{figure}
  \begin{center}
  \tikzset{
    treenode/.style = {shape=rectangle, rounded corners,
                       draw=black, align=center, line width=width},
    specialedge/.style = {line width=1.5pt}
  }

  \begin{tikzpicture}
    [
      grow                    = down,
      sibling distance        = 5em,
      level distance          = 3em,
      edge from parent/.style = {draw=black},
      every node/.style       = {treenode},
      sloped
    ]
    \node [treenode, fill=tealDeer, label = {[label distance = 0.5em]}] (v0) {Conv$(60 \times 7 \times 7)$}
      child {
        child {
          child {
                child {
                  node [treenode, fill=navyPurple] {Classifier}
                  edge from parent
                }
                node [treenode, fill=tealDeer] {BatchNorm}
                edge from parent
           }
          node [treenode, fill=tealDeer] {MaxPool$(3\times3)$}
          edge from parent
        }
        node [treenode, fill=dodgerBlue, label = {[label distance = 0.5em] }] {ReLU} 
        edge from parent
      };

  \end{tikzpicture}

  \end{center}
  \caption[An output example of the algorithm to compute a core stable coalition structure.]{An output example of the algorithm to compute a core stable coalition structure for games represented by forests. Each color represents a coalition.}
  \label{fig:tree}
\end{figure}

To exemplify our method, we create a synthetic dataset of points of two classes. We can see this dataset in Figure \ref{fig:ex-groups}. The points are distributed in two regions, and it is not possible to separate the yellow class linearly from the violet class. To center the points,  we subtracted the mean and divided by the standard deviation. If we look only to violet points, it is clear that they form two clusters. In turn, the yellow dots form a single cluster. We then take the center of each of these three clusters and apply convolution to the set of points using these centers as filters. The result of the convolution after applying the ReLU function can be seen in Figure \ref{fig:ex-groups-after}. For most points after the transformation made by convolution and the ReLU function, only one coordinate was positive, this being the coordinate which value is the internal product with the center of the cluster of which these points are part. These points lie along the three axes of space $\mathbb{R}^3$, each cluster on an axis. The other points are farthest from their cluster center, and the internal product with the center of another cluster is also positive. Cluster centers functioned as good filters to detect similar points, even though they detected points from other clusters.

\section{Experiments}
We use a dataset formed from patches of aerial images made by \cite{8899005} from imagery provided by WeRobotics. The imagery with a resolution of 8cm was captured by the World Bank's UAVs for Disaster Resilience Program, collaborating with WeRobotics and OpenAerialMap in the Kindom of Tonga in October 2017. The Humanitarian OpenStreetMap community provided the annotation. The dataset consists of 13587 patches of dimensions $90 \times 90$ of coconut (10268) and non-coconut (3319). The latter class has a large variability. The problem of interest is to classify the images in one of the two classes. We only report experiments with one dataset due to the space limitation.

As in reality, labeling images is a costly and time-consuming process, and then we assume that we only knew the correct labels of a small set of samples (one hundred images for each class, totaling two hundred images). That way, we would have only those two hundred images to train any models that we would use to try to solve the problem.

To reduce the impact of chance on the results, we randomly selected three sets of 200 images for training and three sets of 11387 images for testing. We wanted to represent a scenario where the network designer did not have much effort to annotate all these images. From each of these sets of training images, we selected four images using the 2D space projection of the 200 images using the t-SNE. We tried to select images that were part of more cohesive groups. Thus, these images are more likely to have characteristics that represent their group.

\begin{figure}[!t]
    \centering
    \subfloat[\label{fig:markers1}]{\includegraphics[width=.3\linewidth]{figures/markers1}
    }
    ~
    \subfloat[\label{fig:markers2}]{\includegraphics[width=.3\linewidth]{figures/markers2}
    }
    \\
    \subfloat[\label{fig:markers3}]{\includegraphics[width=.3\linewidth]{figures/markers3}
    }
    ~
    \subfloat[\label{fig:markers4}]{\includegraphics[width=.3\linewidth]{figures/markers4}
    }
    \caption{Markers used for training. (a) Markers on an image which contains a coconut tree. (b) Markers on an image which does not contain a coconut tree. (c) Markers on an image which does not contain a coconut tree. (d) Markers on an image which does not contain a coconut tree.}
    \label{fig:markers}
\end{figure}

We placed markers on these images in regions that we consider important to identify the coconut and non-coconut (more diverse) classes. In Figure \ref{fig:markers} we can the four images of one the splits. With the markers, we created a convolutional layer with filters of dimensions $7 \times 7$ using the method described in Section \ref{sec:method}. The network architecture can be seen in the Figure. We used a classifier similar to that of VGG-16. We used K-means to cluster the coconut tree patches in 30 clusters and the non-coconut tree patches in also 30 clusters to find the convolutional layer filter. 

\begin{table}[!t]
  \begin{center}
  \begin{tabular}{|l|c|c|c|}
  \hline
   Method & Precision & Recall & F-score \\
  \hline\hline
    I-Ours & $\boldsymbol{0.86 \pm 0.01}$ & $\boldsymbol{0.84 \pm 0.02}$ & $\boldsymbol{0.85 \pm 0.02}$\\
    II-Ours (fine tuned) & $\boldsymbol{0.86 \pm 0.01}$ & $\boldsymbol{0.84 \pm 0.02}$ & $0.84 \pm 0.02$\\
    III-Ours with SVM & $0.836 \pm 0.015 $ & $ 0.793 \pm 0.025$ & $ 0.804 \pm 0.022$\\
    IV-VGG & $0.84 \pm 0.01$ & $0.76 \pm 0.03$ & $0.77 \pm 0.02 $ \\
    V-VGG (fine tuned) & $0.84 \pm 0.01$ & $0.75 \pm 0.03$ & $0.77 \pm 0.02 $ \\
  \hline
  \end{tabular}
  \end{center}
  \caption{Results.}
  \label{tab:results}
\end{table}


We tested three scenarios: (I) using out method to learn the feature extractor weights and the classifier with backpropagation; (II) using our method as initialization and train the whole model with backpropagation; (III) using SVM Linear as classification. For VGG-16, we tested two secenarios: (IV) initializing with Xavier initialization and training from screath; (V) using ImageNet weights and training with the training set images. The results can be seen in Table \ref{tab:results}. As the experiments were repeated three times, we reported the mean and standard deviation of each the metrics in the test set. 

\begin{figure}
  \centering
  \subfloat{\includegraphics[width=.3\linewidth]{figures/qualitative/ours/upper-coco-108.png}
  }
  ~
  \subfloat{\includegraphics[width=.3\linewidth]{figures/qualitative/ours/upper-coco-2473.png}
  }
  ~
  \subfloat{\includegraphics[width=.3\linewidth]{figures/qualitative/ours/upper-non_coco-506.png}
  }
  \\
  \subfloat{\includegraphics[width=.3\linewidth]{figures/qualitative/vgg/lower-coco-1098.png}
  }
  ~
  \subfloat{\includegraphics[width=.3\linewidth]{figures/qualitative/vgg/lower-coco-2400.png}
  }
  ~
  \subfloat{\includegraphics[width=.3\linewidth]{figures/qualitative/vgg/lower-non_coco-105.png}
  }
  \caption{Misclassified images. The first row has images misclassified by the network trained by our method, and the second row has images misclassified by VGG-16. The two first images of a row have coconut trees.}
  \label{fig:ex-classification}
\end{figure}

Our method achieved better precision than conventionally trained VGG, and the difference between precision and recall is much smaller. The Linear SVM did not prove to be a good classifier like the MLP, but, on average, it still got a little better than the VGG-16. Using our method as initialization and training the entire network with backpropagation did not improve the result; this implies that the filters we created are already good filters given the training set. Our feature extractor training method was able to learn good filters from a minimal set of images. VGG has many parameters to learn, so it is susceptible to overfitting with such a small training set. We do not use any technique to handle small data sets to train VGG because we intended to verify our approach viability, and these techniques could also be used with our method. In Figure \ref{fig:ex-classification} we can see examples of misclassified images. In these images, coconut trees appear in different angles, sizes, and shapes, or the boundaries between the coconut tree and the background are tenuous, making it more difficult to identify them. In turn, images that do not contain coconut trees, contain trees that resemble their shape. The network designer could put markers in those images in other to improve the feature extractor.

\begin{figure}
  \centering
  \subfloat[\label{fig:vis-input}]{\includegraphics[width=.3\linewidth]{figures/input}
  }
  ~
  \subfloat[\label{fig:vis-feature-extractor}]{\includegraphics[width=.3\linewidth]{figures/feature_extractor}
  }
  ~
  \subfloat[]{\includegraphics[width=.3\linewidth]{figures/classifier}
  \label{fig:vis-classifier}}
  \caption{This image shows visualizations of the input images and the intermediate layer's outputs. Red points represent background images, and green points represent images that have a coconut tree. The projection from high dimensionality space to 2D was made using t-SNE. The original image shows much overlap between the classes; after applying the feature extractor with only one convolutional layer, we can see that the overlapping decreases considerably. Before the decision layer, the last layer put most of the images from one class on one side and the other on the other side. (a) Visualization of original image feature space. (b) Visualization of the feature extractor's outputs. (c) Visualization of outputs from the last classifier hidden layer.}
\end{figure}

As motivated by Rauber et al. \cite{rauber2016visualizing}, we made projections of three stages of the network using t-SNE \cite{maaten2008visualizing} to understand how the feature extractor trained with our method transforms the image spaces. In Figure \ref{fig:vis-input}, we can see the projection of the original images (in the LAB color space). Red points are images that contain coconut trees, and green points are images that do not contain coconut trees. In the projection in the image of Figure \ref{fig:vis-input}, we see that the images of coconut trees and non-coconut trees are almost entirely overlapping. Coconut tree images are more frequent in some regions but are very dispersed. In the image of Figure \ref{fig:vis-feature-extractor}, we see the projection of the feature extractor output. The coconut tree images are more concentrated, forming some clusters, and it is possible to notice that there is much less overlap. Finally, we see in Figure \ref{fig:vis-classifier} the output of the previous layer of the decision layer of the classifier. The classifier managed to organize conquest and non-coconut images in different regions of space, although there is overlap as the model failed to achieve 100\% accuracy.

\section{Conclusions}

We present a way to train a feature extractor without using backpropagation capable of learning good filters from a minimal set of images. To evaluate the method, we proposed experiments to classify images that contain achievements. Our approach has achieved competitive results with state of the art.

\bibliographystyle{IEEEtran}
\bibliography{bibliography}

\end{document}


