\documentclass[10pt,twocolumn,letterpaper]{article}

\usepackage{cvpr}
\usepackage{times}
\usepackage{epsfig}
\usepackage{graphicx}
\usepackage{amsmath}
\usepackage{amssymb}

% Include other packages here, before hyperref.

\newcommand{\R}{\mathbb{R}}
\newcommand{\A}{\mathcal{A}}
\newcommand{\Z}{\mathbb{Z}}
\newcommand{\m}[1]{\pmb{\mathrm{#1}}}

% If you comment hyperref and then uncomment it, you should delete
% egpaper.aux before re-running latex.  (Or just hit 'q' on the first latex
% run, let it finish, and you should be clear).
\usepackage[pagebackref=true,breaklinks=true,letterpaper=true,colorlinks,bookmarks=false]{hyperref}

% \cvprfinalcopy % *** Uncomment this line for the final submission

\def\cvprPaperID{****} % *** Enter the EV Paper ID here
\def\httilde{\mbox{\tt\raisebox{-.5ex}{\symbol{126}}}}

% Pages are numbered in submission mode, and unnumbered in camera-ready
\ifcvprfinal\pagestyle{empty}\fi
\begin{document}

%%%%%%%%% TITLE
\title{Learning filters from markers}
\author{Italos Estilon de Souza\\
Institute of Computing - University of Campinas (UNICAMP)\\
Institution1 address\\
{\tt\small italos.souza@ic.unicamp.br}
% For a paper whose authors are all at the same institution,
% omit the following lines up until the closing ``}''.
% Additional authors and addresses can be added with ``\and'',
% just like the second author.
% To save space, use either the email address or home page, not both
\and
Alexandre Falcão\\
{\tt\small afalcao@ic.unicamp.br}
}

\maketitle
%\thispagestyle{empty}

%%%%%%%%% ABSTRACT


%%%%%%%%% BODY TEXT
\section{Introduction}

\section{Preliminaries}

Let $D \subset \Z^d$ for $d > 1$ and let $\m{I}: D \to \R^b$, with $b \ge 1$, be a function that assings to each $p \in D$ a vector $\m{I}(p) = (I_1(p), I_2(p), \ldots, I_b(p))$.
A multi-band and multi-dimensional image is a pair ${\hat{I} = (D, \m{I})}$. We say that $D$ is the \textit{image domain}, $\m{I}(p)$ is the \textit{local feature vector} of a space element (\textit{spel}) $p \in D$, $b$ is the number of bands, and $d$ is the number of dimensions. In this work, we focus on images with $d=2$.

An adjacency relation $\A \subseteq D \times D$ is a binary relation between spels $p, q \in D$. We focus on reflexive and translation-invariant relations. More precisely, we focus on adjacency relations given by $\A_{i,j} = \{ (p, q) \in D : \|x_p - x_q\| \le i, \|y_p - y_q\| \le j \}$
{\small
\bibliographystyle{ieee_fullname}
\bibliography{egbib}
}

\end{document}
